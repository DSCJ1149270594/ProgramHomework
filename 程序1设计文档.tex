\documentclass[withoutpreface,bwprint]{cumcmthesis}  %去掉封面与编号页
\usepackage{appendix}
\usepackage{subfigure}	%用于排版多张图片
\usepackage{float}	%用于排版图片位置
\usepackage{pdfpages} % 拼接pdf
\usepackage{url}
\title{\heiti{\zihao{3}身份证号查验程序设计文档}}

\begin{document}
	\includepdf{封面1.pdf}
	\tableofcontents
	\maketitle

	
	%新的一页
	%\newpage
	%从这里开始是第一个大的标题
	\section{题目解析}%一
	\subsection{需求}
	用户需要通过一个程序查验向该程序输入的身份证号码是否有效。
	\subsection{功能概述} 
	身份证号查验程序将验证输入的所有的身份证号码(一次最多输入10000个,按$Enter$键结束当前身份证号输入,按按q或Q结束所有身份证号输入)是否有效,并对所有无效号码进行提示与输出。
	
	\subsection{查验原理}
	一个合法的身份证号码由17位地区、日期编号和顺序编号加1位校验码组成,共计18位。
	\subsubsection{长度和字符}
	程序会先判断输入的身份证号长度是否为18位。如果输入长度不是18位(包括不足或超出18位),程序将提示“身份证号码长度( x位 )不正确,请重新输入”,其中x是检测的当前输入身份证号码的长度,如图(\ref{long})所示。
	\begin{figure}[ht]
		\centering
		\includegraphics[width=.183\textwidth]{LSTM神经网络结构图.png}
		\caption{长短期记忆网络LSTM架构}
		\label{LSTM}
	\end{figure}
	校验码的计算规则如下:
	\begin{itemize}
		\item 先对前17位数字加权求和,权重分配为:{7, 9, 10, 5, 8, 4, 2, 1, 6, 3, 
			7, 9, 10, 5, 8, 4, 2};
		\item 然后将计算的和对11取模得到值Z;
		\item 最后按照表(\ref{Z-M})检查值Z有没有对应的校验码M:
		Z:0 1 2 3 4 5 6 7 8 9 10
		M:1 0 X 9 8 7 6 5 4 3 2
	\end{itemize}
	
	%\vspace{16pt}
	\begin{table}[h] 
		\centering
		\caption{Z值和校验码M值对应关系表} 
		\begin{tabular}{cccccccccccc}  
			\toprule  
			Z & 0 & 1 & 2 & 3 & 4 & 5 & 6 & 7 & 8 & 9 & 10 \\  
			\midrule  
			M & 1 & 0 & 9 & 8 & 7 & 6 & 5 & 4 & 3 & 2 & 1  \\  
			\bottomrule  
		\end{tabular}
		\label{Z-M}  
		
	\end{table}  


	
	
	
	
	\subsection{问题重述}
	导线的温度主要受导线内电流的大小以及环境温度影响,同时也与弧垂角度的大小和导线长度相关。而安装在导线上的两台高架温度线监测设备(OTLM)可以收集温度和弧垂角度等数据。本文需要根据这两台OTLM设备监测的数据建立模型,以解决以下四个问题:
	
	\begin{enumerate}
		\item 用OTLM1设备监测的数据建立模型,并预测2020年8月的动态载流量需求;通过分别间隔0.5小时、1小时、2小时、4小时、8小时、16小时、32小时、64小时的载流量变化情况来分析电流变化具有什么周期性特征。
		
		\item 根据附件所提供的数据分析载流量受弧垂角度和温度的具体影响。
		
		\item 根据附件中OTLM设备输出的关于L型切面的四脚钢结构的架空输电线路的导体温度、环境温度和振动情况等数据,在张力和弧垂角度线性相关的前提下研究导线电流变化和钢架腿部每侧应力变化的相关性。
		
		\item 当2020年8月15日用电量大幅上涨时,计算最大增容量并且研究对应的增容方案。
	\end{enumerate}

        %%%%%%%%%%%%%%%%%%%%%%%%%%%%问题分析%%%%%%%%%%%%%%%%%%%%%%%%%%%%
	%从这里开始是第二个大标题
	%\newpage
	\section{关键数据结构和功能模块设计}
	
	\subsection{问题总分析}
	本文在回答问题前,首先通过归一化、数据类型转换、分类以及引入基于密度的LOF局部异常因子筛除离群点等手段预处理附件中的数据。之后对四个问题进行如下分析:
 
        架空线路载流量的变化受到众多方面因素的影响,使得载流量的预测和增容方案的制定十分复杂。在众多的指标中,本文精确地筛选对载流量影响较大的指标,排除了湿度、经纬度、海拔高度、太阳辐射和风造成的影响,并认为环境温度、导线温度、弧垂角度和应力是影响载流量的主要因素。
        
        本文首先分析了载流量的周期性特征,其次搭建了环境温度、导线温度、弧垂角度和应力与载流量之间的关系,最后本文利用均方误差、拟合优度等方法检验并评估了模型的合理性和泛化能力。
	
	本文的总体问题分析流程图如下:
	
	\begin{figure}[H]
		
		\label{wentifenxi}     %创立一个标签,下文可以直接引用其中的名字,如图\ref{}所示即可
		\centering
		\includegraphics[width=.8\textwidth]{问题分析框图.png}
		\caption{问题分析流程}  %(引用图片)
	\end{figure}

 
	\subsection{具体问题的分析}
 
	\subsubsection{问题一的分析}
	问题一要求选取OTLM1设备的采样数据,建立数学模型预测2020年8月的动态载流量需求,并分析以下
        不同时间间隔30min, 1h, 2h, 4h, 8h, 16h, 32h, 64h的载流量的变化情况,分析电流的周期性特征。
	
	对于预测2020年8月的动态载流量需求,本文使用长短期记忆网络LSTM(Long Short-Term Memory)来分析
        表格中提供的导体电流数据,进而预测2020年8月的导体电流变化,即动态载流量需求。
        
        同时本题还要求分析动态载流量的变化情况,因此,本文再次使用了LSTM循环神经网络来分析表格中提供的导体温度数据,
        以预测2020年8月的导体温度变化,再通过\textbf{5.1 电载量与导体温度的关系}计算出对应的动态载流量。
        
        具体操作步骤如下:
        \begin{enumerate}
            \item 数据预处理:加载数据,合并多个月份的文件,选择相关的特征列。
            \item 数据归一化:将输入的数据进行归一化处理 。
            \item 构建数据集:为LSTM模型准备适当的输入格式。
            \item 构建LSTM模型架构并训练LSTM模型。
            \item 预测未来数据:预测2020年8月的导体数据并进行反归一化,此处分别得到了动态载流量需求预测值和导体温度预测值。
            \item 计算载流量:根据电载量与导体温度的关系,使用第五步得出的导体温度预测值计算出对应的动态载流量。
        \end{enumerate}
        
        对于分析动态载流量的变化情况,本文使用线性趋势估计(需进行显著性检验)和傅里叶分析(计算傅里叶变换FFT,得到数据的频谱图,识别出主要的频率成分);对于分析电流(动态载流量需求)的周期性特征,本文采用时间序列分解:分离出趋势、季节性和残差。
        
	\subsubsection{问题二的分析}
        问题二要求综合考虑OTLM1和OTLM2的采用数据,分析温度、弧垂角度对载流量的定量影响。
        
        由IEEE-738标准,架空裸导线载流量与导体温度成文本第\pageref{calc_I}页式(\ref{calc_I})的函数关系,因此本文利用现有数据,希望经由导体温度来建立载流量与环境温度和弧垂角度的关系。在采用Pearson线性相关分析后,本文引入梯度下降法计算导体温度与环境温度和弧垂角度的线性方程,其相对于正规方程法具有较低复杂度,适合本题大规模数据计算。并基于传统梯度下降,本文引入学习率自适应修改以解决梯度爆炸问题并提高收敛速度。
        本文引入方差膨胀系数计算多重共线性,其相对于矩阵条件系数具有更好的易解释性。
        本文对拟合数据进行基于Cook's距离的离群点诊断,并筛除以获得更好的拟合效果。
        通过分别基于OTLM1和OTLM2的采用数据建立线性方程,本文模型获得了更高的拟合优度。最后基于均方误差对二方程系数加权,可得优化后的线性方程。最后本文引入拟合优度$R^{2}$来衡量模型的表现力。
	
	\subsubsection{问题三的分析}
	问题三要求分析在L型切面脚钢结构中钢架腿部每侧应力变化和导线电流变化的关系。
 
        经分析,题中“电流”一词应等价于电载量$I$,因应力可体现张力的变化,且弧垂角度与张力线性相关,故本文可认为弧垂角度同样与应力线性相关。又由于在本文第二题已求出载流量与弧垂角度的关系,因此本文认为需先求出弧垂角度与应力的具体线性函数,再根据公式变换即可得到导线电流变化与钢架腿部每侧应力变化之间的关系。由于应力数据测量对稳定性有较高要求,本文有必要先依据题目所给的标准偏差以一定比例筛除偏差较大数据。经观察知不同设备测得的应力数据难以直接比较,故本文依据八类应力数据分别建模。由于数据离散度较高,本文在线性拟合前先使用基于密度的LOF局部离散因子筛除离群点。在模型选择上,本文依旧使用收敛较快的自适应修改学习率的梯度下降法。最后,本文使用拟合优度给出模型评估,并由公式变换得到最终结果。\label{p3_ana}
        

        \subsubsection{问题四的分析}
        问题四要求分析在2020年8月15日用电量大幅上涨的前提下的载流量增容方案和最大增容量。

        根据\textbf{5.1 电载量与导体温度的关系}的具体分析,可知在稳态热平衡假设下,
        最大理论载流量与导线的温度有着较强的依赖关系。
        具体到本问题,本文将使用问题一中的LSTM模型预测出2020年8月份的导线温度大致变化情况,
        再根据公式(\ref{I})计算得出对应的最大载流量波动范围,这也就是所求的最大增容量的极限。

        现在讨论在有了该最大增容量的先验条件后,应如何制定合理的增容方案:
        由公式(\ref{I_ta})可得载流量与环境温度和弧垂角度的定量关系。
        在实际情况下,我们是无法干预环境温度的,只能进行预测,
        这里本文同样使用LSTM模型(使用表格中的环境温度数据进行训练)进行预测。
        但弧垂角度可以人为调控,因此可以使用相应的导线应力弧垂计算软件分析现有弧垂对线路的影响,
        并根据相关规程和标准,制定相应的调整方案。
        除此之外,还可以应用动态线路额定(Dynamic Line Rating, DLR)技术,增加导线截面积,
        使用高温超导材料,优化运行模式等方式来进行增容。
 
	%第三部分开始了

	\newpage
	\section {调试与测试}
 
	\subsection {电载量与导体温度的关系}
        
        电载量是本题讨论的核心物理量,其与外界环境及导体本身性质构成复杂的依赖关系。虽然题目并未给出直接的电载量数据,但根据电气电子工程师协会IEEE-738标准的电载量计算公式\textsuperscript{\cite{ref1}}
        可进行相关求解,本题后续将基于该标准计算载流量。

        为标准化问题研究,本文取题中架空输电线路导线为国家标准GB 1179-83\textsuperscript{\cite{ref2}}
        下的70$mm^{2}$钢芯铝绞线,结构数为6(铝)/1(钢),铝线导电率为61.2\%IACS。下文式(\ref{R}、\ref{R0})中常数$\alpha $、$\rho _{20}$、$\lambda_{am}$的具体值即在上述前提下取得。
        
        基于IEEE-738标准,本文可以在稳态热平衡的前提下根据如下公式
        \begin{equation}
            q_{c}+q_{r}=q_{s}+I^2\cdot R(T_{avg})
        \end{equation}
        
        推出电载量
        \begin{equation}
            I=\sqrt{\frac{q_c+q_r-q_s}{R\left( T_{avg} \right)}}=\sqrt{\frac{q_c+q_r-q_s}{k\times R_{\text{直}}(T_{avg})}}
            \label{calc_I}
        \end{equation}
        
        式(\ref{calc_I})中,$k$为交流电阻与直流电阻的比值,是一个与电流磁场有关的近似于1的数值。为集中体现载流量与导体温度等题设变量关系,本文取$k$为1。$R_{\text{直}}(T_{avg})$是温度为$T_{avg}$时导体的直流电阻。$T_{avg}$在文献\cite{ref1}中的标准定义为铝绞线层的平均温度,本文认为它和题目所给观测$Line\:Tempture$(导体温度)等价。$R_{\text{直}}(T_{avg})$可由如下公式计算
        \begin{equation}
            R_{\text{直}}(T_{avg})=R_0\times \left( 1+\alpha \times \left( T_{avg}-T_0 \right) \right)  
            \label{R}
        \end{equation}
        其中,$R_0$为温度为$20^\circ$C时导体的直流电阻,$\alpha$为摩尔根实验下测得的电阻温度系数$0.00466/^\circ$C,$T_0$为$20^\circ$C。
        
        式(\ref{R})中$R_{0}$可计算如下
        \begin{equation}
            R_0=\frac{\rho _{20}\times \lambda _{am}}{S}
            \label{R0}
        \end{equation}
        其中$\rho _{20}$为温度为$20^\circ$C时的铝线电阻率0.028172$\varOmega \cdot mm^2/m$,$\lambda _{am}$为铝线股的平均绞入系数1.0152,$S$为导线的总横截面积70mm$^{2}$

        式(\ref{calc_I})中,单位长度的对流散热率$q_{c}$、单位长度的辐射散热率$q_{r}$以及来自太阳的热量获取率$q_{s}$在稳态热平衡假设下可视为不变。由IEEE-738基于国际标准下太阳热通量、太阳赤纬和太阳高度的计算方法得到的$q_{c}$、$q_{r}$和$q_{s}$值为
        \[q_{c}=81.93 W/m,\;\;\;q_{r}=39.1 W/m,\;\;\;q_{s}=22.44 W/m \]

        因此,在本文所设基本前提和假设下,整合式(\ref{calc_I}、\ref{R}、\ref{R0})并带入相关常量值,可得电载量与导体温度的函数关系
        \begin{equation}
            I=\sqrt{\frac{6901.0}{(0.0001333\times T_{avg} + 0.02593)}}
            \label{I}
        \end{equation}

        所以,本文希望经由导体温度,讨论载流量与题设变量的关系。
        
        
	\subsection {创建LSTM模型}

        本例的时间序列数据时间跨度大,总体数据多,建立的模型需要克服处理长序列时梯度消失或梯度爆炸的问题。而LSTM作为一种特殊的RNN,就能很好地解决这个问题。LSTM通过引入一种特殊的记忆单元,使得模型能够在不损失长期记忆的情况下,更好地处理序列数据。具体来说,LSTM通过引入三个门(遗忘门、输入门和输出门)来控制信息的传播,从而实现对长期依赖关系的有效捕获。以下是具体操作流程:  

        \subsubsection{数据收集和预处理}
                \begin{itemize}
                    \item 数据来源:2019-7到2020-7两个装置收集到的导线电流数据(间隔为10分钟)。
                    \item 数据大小:43207个样本,每个样本有2个特征(对应两个设备)。
                    \item 数据预处理:合并多个月份的文件,选择相关的特征列,
                          然后使用MinMaxScaler这个归一化器将输入的数据进行归一化处理,
                          再转换为DataFrame格式(方便后续处理),忽略缺失值。
                \end{itemize}
                
        \subsubsection{数据集创建}
                设置数据集的时间窗口大小为50(模型在进行计算或分析时所考虑的数据点数量),再调整数据维度至 $(samples, time steps, features)$ 以适应LSTM输入要求,接着设定数据集的 $80\%$ 用于训练, $20\%$ 用于验证,设置验证集的目的是计算Loss值来判定模型的准确度。
                经过设置后,训练集形状为 $(34565, 50, 2)$ ,验证集形状为 $(8642, 50, 2)$ 。
                其中,第一个数字表示训练集或验证集中的样本数量,第二个数字表示每个样本的时间步长,第三个数字表示每个时间步长对应的特征数量。
                
        \subsubsection{建立LSTM神经网络架构}
        本文使用 Sequential 模型(Keras中用于构建神经网络的一种模型)。
        Keras 的 Sequential 模型是一个线性的层次堆栈,
        可以通过传递一系列层实例给构造器来创建一个序列模型。
        这种模型可以构建非常复杂的神经网络,包括全连接神经网络、
        卷积神经网络(CNN)、循环神经网络(RNN)等。
        Sequential模型的主要特点是网络只有一个输入和一个输出,且该网络是层的线性堆叠。

        以下是本文所使用的LSTM模型的结构说明:
            \begin{enumerate}
                \item \textbf{输入层}:LSTM层接受形状为(时间步长, 特征数)的输入数据。
                        这意味着每个输入样本由50个时间步组成,每个时间步包含2个特征。
                \item \textbf{第一个LSTM层}:包含100个LSTM单元,激活函数为 $\tanh$。
                        输入数据通过此层后,将进行非线性变换,并提取出时间序列中的高级特征。
                        返回其所有时间步的隐藏状态组成的序列(形状为 $(None, 50, 100)$ 的张量)到下一层。
                \item \textbf{Dropout层}:包含20\%的dropout率,
                        用于在训练过程中随机将网络中的一部分节点“关闭”,以减少过拟合现象。
                \item \textbf{第二个LSTM层}:包含50个LSTM单元,激活函数同样为 $\tanh$ 。
                        此层将接受上一个LSTM层输出的每个时间步的隐藏状态作为输入。
                        经过此层后,模型将再次提取特征,并输出最后一个时间步的隐藏状态,作为整个序列的压缩表示。
                \item \textbf{全连接层(Dense层)}:包含一个具有两个神经元的全连接层,用于输出最终的预测结果。
                        由于模型的目标是预测两个连续的特征,因此输出层的神经元数量设置为2。
            \end{enumerate}

            在建立的LSTM模型中,共有71,502个总参数,其中71,502个可训练参数,0个不可训练参数。
                        %\vspace{16pt}
                        \begin{table}[h] 
                            \centering
                            \caption{模型体系结构图表} 
                            \begin{tabular}{ccccc}  
                                \toprule  
                                    Layer (type) & Output Shape & Param \#   \\  
                                \midrule  
                                    lstm (LSTM) & (None, 50, 100) & 41200    \\  
                                    dropout (Dropout) & (None, 50, 100) & 0  \\  
                                    lstm\_1 (LSTM) & (None, 50) & 30200      \\  
                                    dense (Dense) & (None, 2) & 102          \\  
                                \bottomrule  
                            \end{tabular}  
                             
                        \end{table}  
                        
        本文建立的LSTM网络架构如图(\ref{LSTM})所示:                
        \begin{figure}[ht]
		  \centering
		  \includegraphics[width=.183\textwidth]{LSTM神经网络结构图.png}
		  \caption{长短期记忆网络LSTM架构}
            \label{LSTM}
        \end{figure}
 
        在两个LSTM层中的LSTM单元如图(\ref{lstm})所示:
        \begin{figure}[ht]           
		  \centering
		  \includegraphics[width=1.0\textwidth]{LSTM单元2.png}
		  \caption{LSTM单元}
            \label{lstm}
	\end{figure}
                        
        \subsubsection{LSTM模型的训练}
            RMSprop是一种自适应学习率的优化算法,适用于非凸优化问题,能够自动调整每个参数的学习率。本文使用RMSprop作为优化器,设置学习率为0.001,并使用均方误差$(Mean Squared Error, MSE)$作为损失函数,以衡量模型预测值与真实值之间的差异。
            模型在训练集共训练50个周期$(epochs)$。
            在每个周期中,模型将使用大小为32的批次$(batch size)$进行迭代训练,
            并在训练数据集的20\%上进行验证,以监控模型的性能。

        总的来说,本文建立了一个基于LSTM网络的模型,用于处理时间序列数据并预测两个连续的特征。模型采用两层LSTM结构,通过堆叠多个LSTM层来提取输入序列中的高级特征,并使用Dropout层来减少过拟合。模型使用RMSprop优化器和均方误差损失函数进行编译和训练,并最终在验证集上进行性能评估和模型检验(模型评估和检验详见\textbf{5.6 LSTM模型的评估和检验})。

	\subsection {使用LSTM模型进行预测}
	      使用上文模型来预测2020年8月的导线电流变化,即动态载流量需求,
            期间还需要进行数据的反归一化,
            预测的结果预设置为一个月两个设备分别对应的数据(频率为10min)。
            因为数据量庞大,故对预测结果进行重采样
            (本例使用mean作为聚合函数,当然也可以使用其他聚合函数,如sum, min, max等)。
            将重采样的数据进行绘图,得到如下的结果:
            
        \begin{figure}[htbp] 
            \begin{minipage}{.48\textwidth} % 左侧图像占文本宽度的48%  
                \centering  
                \includegraphics[width=\linewidth]{8月份设备1电载量需求预测图.png}  
                \caption{2020\_8\_设备1载流量需求预测图}  
            \end{minipage}  
            \hfill % 填充空白,使得两个图像并排显示  
            \begin{minipage}{.48\textwidth} % 右侧图像占文本宽度的48%  
                \centering  
                \includegraphics[width=\linewidth]{8月份设备2电载量需求预测图.png}  
                \caption{2020\_8\_设备2载流量需求预测图}  
            \end{minipage}  
        \end{figure}

        同理,本文使用相同的模型架构和相同的训练方法对2019-7到2020-7两个装置收集到的导线温度数据(间隔为10分钟)也进行LSTM模型创建,并进行2020年8月的导线温度的预测,再根据电载量与导体温度的关系计算出对应的动态载流量。也对预测结果进行重采样,再进行绘图,得到如下的结果:

        \begin{figure}[htbp] 
            \begin{minipage}{.48\textwidth} % 左侧图像占文本宽度的48%  
                \centering  
                \includegraphics[width=\linewidth]{8月份设备1电载量预测图.png}  
                \caption{2020\_8\_设备1载流量预测图}  
            \end{minipage}  
            \hfill % 填充空白,使得两个图像并排显示  
            \begin{minipage}{.48\textwidth} % 右侧图像占文本宽度的48%  
                \centering  
                \includegraphics[width=\linewidth]{8月份设备2电载量预测图.png}  
                \caption{2020\_8\_设备2载流量预测图}  
            \end{minipage}  
        \end{figure}
        
        \subsection {分析不同时间间隔下的动态载流量的变化情况}
        
        \subsubsection{线性趋势估计和显著性检验}
            现在本文从数学和统计的角度分析上文所预测的动态载流量的变化趋势。

            \vspace{12pt}

            \textbf{线性趋势估计}:使用线性回归模型拟合数据。假设趋势可以表示为:
                \begin{equation}
		          T(x)=\beta_0+\beta_1 x+\epsilon
	        \end{equation}
                
                其中 $x$ 是时间,$\beta_0$ 和 $\beta_1$ 是回归系数,而 $\epsilon$ 代表误差项。
                                 
                使用最小二乘法(OLS)进行估计,该方法试图最小化实际观测值与模型预测值之间的平方误差和:
                \begin{equation}
                    \text{SSE} = \sum_{i=1}^{n} (T_i - \hat{T}_i)^2
                \end{equation}
                
                其中:$\text{SSE}$ 是残差平方和,$n$ 是观测值的数量,$T_i$ 是第 $i$ 个观测值,$\hat{T}_i$ 是第 $i$ 个观测值的模型预测值。

                通过求解使 $\text{SSE}$ 最小的 $\beta_1$ 值,得到线性回归模型的参数估计:
                线性趋势系数 $\beta_1$ 约为 -0.00004,意味着单位时间电流会下降大约 0.00004 单位;截距为 499.2788,本例中它对应月初的载流量值。
                
                判断模型优良的决定系数 $R^2$ (或称为确定系数、拟合优度)的求解公式为:
                \begin{equation}
                    R^2=\frac{\mathrm{SSR}}{\mathrm{SST}}=1-\frac{\mathrm{SSE}}{\mathrm{SST}}
                    \label{R2}
                \end{equation}
                \begin{equation}
                    \mathrm{SSR}=\sum_{i=1}^n\left(\hat{T}_i-\bar{T}\right)^2
                \end{equation}                
                \begin{equation}
                    \operatorname{SST}=\sum_{i=1}^n\left(T_i-\bar{T}\right)^2
                \end{equation}

                其中:$\text{SSR}$ 是回归平方和,$\text{SST}$ 是总平方和,$\text{SSE}$ 是残差平方和,$n$ 是观测值的数量,$T_i$ 是第$i$ 个观测值,$\hat{T}_i$ 是第 $i$ 个观测值的模型预测值,$\bar{T}$ 是观测值的均值。

                本例求得的决定系数 $R^2$ 非常低,约为 0.0001,表明时间变量几乎不能解释动态载流量变化的方差,这意味着该线性模性对于本例不适合。

            \vspace{12pt}
            
            \textbf{趋势的显著性检验}:
                接下来,为了更准确地评估线性趋势的显著性,本文将进行t检验。
                这将帮助本文判断所估计的斜率(趋势系数 $\beta_1$)是否显著地不同于零。

                在线性回归模型中,斜率(或称为回归系数)的t检验统计量定义为:
                \begin{equation}
                    t=\frac{\hat{\beta}_1-\beta_1}{S E\left(\hat{\beta}_1\right)}
                \end{equation}

                其中:$\hat{\beta}_1$ 是斜率的估计值(即样本斜率),$\beta_1$ 是假设的斜率值(在线性趋势的显著性检验中,设它为0),$S E\left(\hat{\beta}_1\right)$ 是斜率估计值的标准误差。

                斜率估计值的标准误差 $S E\left(\hat{\beta}_1\right)$ 由以下公式给出:
                \begin{equation}
                    S E\left(\hat{\beta}_1\right)=\frac{\sqrt{\frac{1}{n-2} \sum_{i=1}^n\left(T i-\hat{T}_i\right)^2}}{\sqrt{\sum_{i=1}^n(x i-\bar{x})^2}}
                \end{equation}

                其中,$n$ 是样本大小,$T_i$ 是第 $i$ 个观测值的因变量,$\hat{T}_i$ 是第 $i$ 个观测值的预测值(即根据回归线得到的预测值),$x_i$ 是第 $i$ 个观测值的自变量,$\bar{x}^2$ 是自变量的样本均值。

                最后,根据以上公式计算出的t统计量和相应的自由度n−2,可以查找t分布表或使用统计软件来确定t统计量对应的p值,从而判断斜率是否显著不为零。
                
                在进行斜率的显著性t检验后,得到了一个接近零的t统计值-0.283,和一个较高的p值0.777,这表明线性趋势的斜率与零没有显著的统计差异。这意味着在统计上,本文没有足够的证据表明存在一个显著的线性趋势。这与之前的 $R^2$ 值很低的观察结果一致,表明\textbf{线性模型不适合描述这组数据}。


        \subsubsection{傅里叶分析}
        现在本文再从数学和统计的角度进一步分析动态载流量的周期性特征。

        \vspace{12pt}

        从上文的动态载流量预测图中可以看出,它表现出明显的波动,这些波动可以通过傅里叶变换分析来进一步量化其频率和振幅,即使用如下形式表示:
        \begin{equation}
            S(t)=\sum_{k=1}^K\left(a_k \cos \left(2 \pi f_k t\right)+b_k \sin \left(2 \pi f_k t\right)\right)
        \end{equation}

        其中,$f_k$ 是特定的频率成分,$a_k$ 和 $b_k$ 是对应的振幅。
        
        周期可以通过计算季节分量中的主要频率成分来确定。

        计算傅里叶变换FFT,得到数据的频谱图如下:

        \begin{figure}[h!]
		  \centering
		  \includegraphics[width=.7\textwidth]{FFT.png}
		  \caption{频谱分析结果}
	\end{figure}

        频谱图显示,数据的主要频率成分集中在低频区域。具体来说,从0$Hz$到约1$Hz$之间,频谱幅值$Y[Freq]$从最大值逐渐减小至接近于零的水平,表明信号中包含了一个较强的低频分量,可能与信号的基本周期性或趋势有关;高频周期性成分不是特别强,说明它的变化主要受到较慢变化趋势的影响,而不是明显的日常或小时级的周期性波动。总的来说,这表明动态载流量的变化是长周期的,在较短的时间内没有太大变化,这与实际情况下,温度变化也是长周期的特点相对应。


        \subsection {分析不同时间间隔下电流(动态载流量需求)的周期性特征}
        \label{model_1}
        时间序列分解用于理解时间序列数据,其目的是将时间序列分解为几个不同的组成部分(通常分解为趋势、季节性和残差三个部分),以便理解其模式和结构。趋势:可以是上升、下降或平稳,反映了数据随时间变化的长期方向。季节性:是时间序列中周期性变化的一种,可以在时间序列中重复出现。残差:是时间序列中除趋势和季节性之外的剩余部分,通常被认为是随机或不可预测的。

        以下是本文应用该方法的具体流程:首先进行数据预处理,将表格中的内容转换为适当的数据类型,将日期设置为索引,重采样(使用mean作为聚合函数)。再提取设备1的电流数据进行时间序列分解。不同时间间隔分解后的结果图详见
        \textbf{附录A第一小节}。
       
        现在对每张结果图中的四个部分进行解释:第一个部分是电流随时间的变化图,后三个部分分别是分解出的趋势图,季节性变化图和残差图。在这八张图中,不难看出后三个部分的数据都十分相似,因此本文重点分析时间间隔为30min的时间序列分解图。
        
        \begin{itemize}
            \item \textbf{趋势部分}:
                电流虽有起伏,但整体上呈现出轻微的上升趋势,这表明电流需求随时间逐渐增加;这种长期趋势可能由多种因素引起,例如工厂的设备使用数量增加,家庭用电增多;长期趋势的理解对于进行设备维护规划和能源需求预测具有重要价值。
            \item \textbf{季节性部分}:
                显示了明显的周期性波动,表明电流载流量在一天中有固定的波峰和波谷,这些变化每天都重复出现,显示出高度的规律性;这种模式可能与工厂的班次作业、商业设施的营业时间或家庭的日常生活有关。例如,电流的峰值可能对应于工作日的开始和结束,低谷可能发生在夜间;通过分析季节性分量,管理者可以优化电力使用,例如调整高需求时段的电力供应,或实施需求响应策略。
            \item \textbf{残差部分}:
                显示了在移除前两者影响之后,电流载流量的随机波动。这些波动较小,说明大部分的变化已经被前两者解释;残差的稳定性表明模型已较好地捕获了数据中的主要模式,剩下的波动可能由不可预测的外部因素或是测量误差引起。如设备故障、意外的电力外部干扰或操作错误;尽管残差看似随机,但这些数据可以帮助识别出正常模式之外的异常情况,对于预防维护和故障诊断尤为重要。
        \end{itemize}
        
        总体而言,动态载流量需求虽有起伏,但整体上呈现出轻微的上升趋势;电流在一天中有固定的波峰和波谷,这些变化每天都重复出现,显示出高度的规律性。这些分析为工业设施管理和优化运营方面提供了重要信息。周期性特征的识别尤其有助于预测未来的负载需求和调整维护计划。

	\subsection {LSTM模型的评估和检验}
        模型在验证集上的性能通过测试损失进行评估,测试损失使用均方误差作为度量指标,
        较低的测试损失表示模型在未见过的数据上具有更好的泛化能力。
        本例针对导线电流数据集的测试损失约为 0.0061,
        本例针对导线温度数据集的测试损失约为 0.0002,
        表明本例模型在验证集上的预测结果与真实标签之间的差异非常小,
        意味着模型的性能较好,能够较准确地预测测试集上的数据

        
	%%%%%%%%%%%%%%%%%%%%%%%%%%%%% 问题二 %%%%%%%%%%%%%%%%%%%%%%%%%%%
	\section{问题二模型的建立与求解}
	\subsection{数据分析}
	\subsubsection{线性相关检验}
	按照本文在第\pageref{I}页的推导,载流量$I$与与导体温度$T_{avg}$呈式(\ref{I})关系。因此,为求解本题,本文需要讨论$T_{avg}$与环境温度$T_{e}$和弧垂角度$A$的二元函数关系。为观察$T_{avg}$相对于$T_{e}$和$A$的变化趋势,本文先分别考虑因变量$T_{avg}$对两个自变量的关系。

        绘制散点图$T_{avg}\sim T_{e}$与$T_{avg}\sim A$,如图(\ref{ggnld})所示。
	\begin{figure}[H]
		\centering
		\includegraphics[width=.7\textwidth]{二元数据分布散点图.png}
		\caption{二元数据分布散点图}  %(引用图片)
		\label{ggnld}     %创立一个标签,上下文可以直接引用其中
	\end{figure}
        由图可知$T_{avg}$与$T_{e}$呈较强线性关系,$T_{avg}$与$A$在分段区间上呈线性关系。
        
        本文引入Pearson系数分析$I$与$T_{avg}$的线性相关性。设$l_{x x},\;l_{y y},\;l_{x y}$分别表示变量$x,\;y\;,\;\;x$与$y$的离均差平方和,相关系数$r$计算如下
	\begin{equation}\label{r}
		r=\frac{\sum(x-\bar{x})(y-\bar{y})}{\sqrt{\sum(x-\bar{x})^{2} \sum(y-\bar{y})^{2}}}=\frac{l_{x y}}{\sqrt{l_{x x} l_{y y}}}
	\end{equation}
         其中
	\begin{align}
		l_{x x}&=(x-\bar{x})^{2}=\sum x^{2}-\frac{\left(\sum x\right)^{2}}{n} \\
        l_{Y y}&=(y-\bar{y})^{2}=\sum y^{2}-\frac{\left(\sum y\right)^{2}}{n} \\
        l_{x y}&=\sum(x-\bar{x})(y-\bar{y})=\sum x y-\frac{\left(\sum x\right)\left(\sum y\right)}{n}
	\end{align}

        这里,相关系数$r$是一个无量纲数,且$-1 \leq r \leq 1$,相关程度$\propto\left|r\right | $
        
        将题目数据带入式(\ref{r})解得$T_{avg}$与$T_{e}$相关系数为
        \[
		r_{ae}=0.9897
	\]
        得出$r_{ae}$接近1,线性相关性强。

        $T_{avg}$与二自变量关系可视作空间曲面,且由上述分析知,该曲面上与$T_{avg}\sim T_{e}$坐标面平行的截线在该坐标面上的投影接近直线。因此考虑用空间平面拟合$T_{avg}$与二自变量关系的关系,即$T_{avg}$与$T_{e},\:A$成二元线性相关。
	\subsubsection{多重共线性检查}
	多重共线性(MTC)指自变量之间成较强线性相关,进行MTC检查的原因有以下两点:

        \textbf{1) MTC 与线性回归模型无完全共线性基本假设矛盾,会导致模型不能反映自变量与因变量的真实影响关系。}
        
	\textbf{2) MTC 会因变量自由度问题致使回归模型不稳定,使模型在拟合优度检验上出现拟合优度统计量偏大。}	
 
	本文引入方差膨胀因子$VIF$衡量多重共线性。
        设 $R_{i}$  为第  $\mathrm{i}$  个变量  $x_{i}$  与其他全部变量  $x_{j}(i=1,2, \ldots, k ; i \neq j)$  的复相关系数,方差膨胀因子$VIF_{i}$表示为
        \begin{equation}\label{eq6}
        VIF_{i}=\frac{1}{1-R_{i}^{2}}
        \end{equation}
        当$VIF_{i}<10$时,不存在多重共线性。
        
        $R_{i}$在数学上等于拟合优度$R^{2}$的算术平方根。而$R^{2}$的计算公式已经在第\pageref{R2}页的式(\ref{R2})中给出。
        
        将$T_{e}$与$A$数据带入式(\ref{eq6}),得膨胀因子向量
        \[
        VIF=\begin{bmatrix}1.0018\\1.0018\end{bmatrix}
        \]
        故自变量$T_{e}$与$A$之间\textbf{不存在多重共线性}。
	
	\subsection{模型建立}
 
        \subsubsection{算法选择}
        \label{grad}
        有别于传统线性回归,本文引入自适应修改学习率的梯度下降法拟合$T_{avg}$与$T_{e}$和$A$的三元线性方程,主要基于以下考虑:
        
        \textbf{1) 线性回归任务中,传统的基于最小二乘法的正规方程法具有$O(3)$复杂度。相较下,基于差分的梯度下降具有$O(1)$复杂度,其在本题大规模的数据集上拥有更快的训练速度。}
        
	\textbf{2) 在梯度下降中引入自适应修改学习率机制,可有效加快模型收敛,并解决了梯度爆炸问题,提高了参数稳定性。}	

        \subsubsection{模型定义}
        \label{model_def}
        在损失函数的选择上,本文进行了诸多考虑。传统线性回归在许多任务上(例如房价预测问题)使用基于均方根误差的对数变换的损失函数($\;\sqrt{\frac{1}{n}\sum_{i=1}^{n} (\ln{y_i} -\ln{\hat{y_i} })^{2} }\;$),即通过残差相对于真实值的比例来衡量模型的拟合效果。但本文认为这类方法不适用于本题,因为该类方法适用的前提是因变量方差呈现随因变量增长而不断增加的非正态分布,而从图(\ref{ggnld})可见本题因变量数据分布并不符合上述条件,从下文残差-频率直方图(第\pageref{res}页图(\ref{res}))中也可见这一情况。所以,本文引入平方误差作为模型损失函数,而非传统线性回归中大量使用的基于均方根误差的对数变换的损失函数。本文引入的平方误差损失函数公式为
        \begin{equation}
        L(X;\theta )=\frac{1}{n}\sum_{i=1}^{n}(x_{i}\theta-y_{i})^2
        \label{sq_loss}
        \end{equation}
        
        在实现梯度下降法时,需计算每一个$\theta_{j}$(维度)下损失函数的梯度,即偏导数
        \begin{equation}
        \frac{\partial L(X;\theta )}{\partial \theta_{j}}=\frac{2}{n} \sum_{i=1}^{n}(x_{i}\theta -y_{i})x_{ij},\;\;\;\;j=1,...,m
        \end{equation}

        改为向量化表示,得到损失函数的梯度向量
        \begin{equation}
        \nabla L(X;\theta )=\begin{pmatrix} \frac{\partial L(X;\theta )}{\partial \theta_{1}}\\ \vdots \\ \frac{\partial L(X;\theta )}{\partial \theta_{m}}\end{pmatrix}=\frac{2}{n} X^{T} (X\theta-y)
        \end{equation}

        得到梯度向量后,每次迭代时以学习率$\eta $调整参数,即
        \begin{equation}
        \theta ^{next}=\theta -\eta \nabla L(X;\theta)
        \end{equation}

        本文引入自适应修改学习率,根据训练损失的变化动态调整学习率,实现前期学习率逐步增大,加速收敛,后期学习缓慢减小,避免跳出局部最优。
        
        本文中学习率变化率具体如表(\ref{tab:lr})所示
        \vspace{16pt}
        \begin{table}[h] 
            \centering
            \caption{模型学习率动态变化率} 
            \begin{tabular}{ccccc}  
                \toprule  
                    训练损失下降时学习率变化率 & 训练损失增大时学习率变化率  \\  
                \midrule  
                    1.2 & 0.8   \\  
                \bottomrule  
            \end{tabular}  
            \label{tab:lr}
        \end{table} 
        
	\subsection{模型训练}
        \subsubsection{训练数据分批处理}
        本文在第二小题的讨论旨在借助于导体温度$T_{avg}$分析载流量$I$与$T_{e}$和$A$的关系。本文认为在此题求解中应当分别考虑设备一和设备二数据,建立两个三元线性方程,最后基于均方误差对两个线性方程赋以不同权重,累加得到优化后的线性关系。主要由于考虑到$T_{avg}$与环境数据成复杂依赖关系,而测得的环境数据因设备不同可能会产生较大误差;并且仅有两个设备,无法通过传统的基于大规模数据特征呈正态分布的统计原理获得与其他影响因素无关的一般规律。

        \subsubsection{训练过程}
        考虑到本文引入的自适应修改学习率的梯度下降法收敛较快,本文给模型设置较低的初始学习率和迭代次数。同时,本文赋予模型对应的二元线性函数全零初始系数,这便于训练损失曲线趋势的呈现,使离群点更易被发现。为实现不同设备数据的训练结果可对比,本文给两个设备对应的模型赋以相同的初始化参数。具体的模型初始化参数如表(\ref{tab:model_param})所示
        \vspace{16pt}
        \begin{table}[h] 
            \centering
            \caption{模型初始化参数} 
            \begin{tabular}{ccccc}  
                \toprule  
                    学习率 & 迭代次数 & 二元线性函数系数 & 最大误差限\\  
                \midrule  
                    $10^{-7}$ & 100 & $\begin{bmatrix}0\\0\\0\end{bmatrix}$ & $10^{-8}$ \\ 
                \bottomrule
            \end{tabular}  
            \label{tab:model_param}
        \end{table} 

        分别将两个设备数据投入模型训练,在完成全部迭代周期后获得训练损失曲线,如图(\ref{loss})
        \begin{figure}[H]
		\centering
		\includegraphics[width=.7\textwidth]{设备一二训练损失.png}
		\caption{设备一、二训练损失曲线}  %(引用图片)
		\label{loss}     %创立一个标签,上下文可以直接引用其中
	\end{figure}

        设$i$为设备号,设备一、二最后一次迭代的训练损失分别为
        \[
        loss_{1}=2.1153,\;\;\;\;loss_{2}=1.6614
        \]
        
        可以看到在迭代初期,一、二设备的训练损失下降得并不平滑,这可能是训练数据的离群点所导致的。因此本文需额外考虑离群点对模型二元线性函数拟合效果的影响。

        本文引入Cook's距离来衡量数据点的离群程度。Cook's距离是拟合值变化的缩放值,其展示了每个观测对拟合响应值的影响。Cook's距离大于平均Cook's距离三倍的观测值可能是异常值。每个Cook's距离$D$中的元素是由于删除一个观测值而导致的拟合响应值的归一化变化。观测$i$的Cook's距离是
        \begin{equation}
        D_{i}=\frac{\sum_{j=1}^{n}(\hat{y}_{j}-\hat{y}_{j(i)})^{2}}{pMSE} 
        \label{cook}
        \end{equation}
        其中
        
        \text{$\bullet\;\;$$\hat{y} _{j}$是第$j$个拟合响应值。} 
        
        \text{$\bullet\;\;$$\hat{y} _{j(i)}$是第$j$个拟合响应值,其中拟合不包括观测$i$。}
        
        \text{$\bullet\;\;$$MSE$表示均方误差。}

        相对于依赖大量专家经验的孤立森林和局部离群因子算法,Cook's距离的评价更为客观和稳定。对于含较少专家经验数据的本题,Cook's距离更加适用。

        将设备一、二训练数据及模型预测值带入式(\ref{cook}),计算Cook's距离后绘制距离分布散点图并划出离群点诊断的距离阈值(平均Cook's距离的三倍),得图(\ref{cook_dis})
        \begin{figure}[H]
		\centering
		\includegraphics[width=.7\textwidth]{设备一二cook距离.png}
		\caption{Cook's距离-观测顺序图}
		\label{cook_dis}     %创立一个标签,上下文可以直接引用其中
	\end{figure}

        由Cook's距离的算法实现知,Cook's距离大于三倍均值的点可能为离群点,这在本题设备一、二观测中的比例分别约为$\;\frac{1}{14}\;$($\;\frac{2834}{40404}\;$)和$\;\frac{1}{15}\;$($\;\frac{2604}{40404}\;$)。本文将上述观测视为离群点并剔除,接着使用剔除后的数据再进行模型训练,在相同的最大迭代次数下获得了更低的训练损失,如下:
        \[
        loss_{new1}=1.2573,\;\;\;\;loss_{new2}=1.0448
        \]

        本文在上述求解中获得了设备一、二数据对应的两个线性函数,接下来需将两个函数系数融合得到优化后的关系式。本文认为最后一次迭代的均方误差(即平方损失)可以反映训练数据的稳定程度和拟合效果,又由于设备一、二的起始训练样本数相同,故本文无需考虑训练样本数差异导致的模型置信度问题。本文通过均方误差来加权设备一、二的模型系数。设$loss_{newi}$为设备$i$的训练损失,$p_{ij}$为设备$i$的模型的第$j$个参数,则权重向量为
        \begin{equation}
        W=\begin{bmatrix}loss_{new2}\\loss_{new1}\end{bmatrix}\oslash\begin{bmatrix}loss_{new1}+loss_{new2}\\loss_{new1}+loss_{new2}\end{bmatrix}=\begin{bmatrix}0.4538\\0.5462\end{bmatrix}
        \end{equation}
        进而得到优化后的$T_{avg}$与$T_{e}$和$A$的二元线性关系为
        \begin{align}\label{TTA}
            T_{avg}=\begin{bmatrix}1 & T_{e} &A\end{bmatrix}\cdot 
            (\begin{bmatrix}p_{11}& p_{21}\\p_{12}& p_{22}\\p_{13}&p_{23}\end{bmatrix}\cdot W)\\=0.01008+1.0174\times T_{e}+0.09953\times A
        \end{align}

	\subsection{关系转换与模型评估}
        本文通过拟合优度来衡量模型表现力。合并剔除离群点后的设备一、二的观测,得到综合数据。由本文第\pageref{R2}页式(\ref{R2})可计算得到模型在综合数据上的拟合优度为
        \[R^{2}=0.9848\]

        由拟合优度的算法原理知,当$R_{2}>0.9$时,得到的线性关系拟合得很好。可见,本文得到的线性模型可以很好地表现$T_{avg}$与$T_{e}$和$A$的二元线性关系。同样,通过图(\ref{liner_result}、\ref{res})我们可以看到本文线性模型的拟合效果,以及呈正态分布的残差(这与本文在第\pageref{sq_loss}页的分析相符)。
        \begin{figure}[h!] 
            \begin{minipage}{.48\textwidth} % 左侧图像占文本宽度的48%  
                \centering  
                \includegraphics[width=\linewidth]{线性模型拟合结果.png}  
                \caption{线性模型拟合三维视图}  
                \label{liner_result}
            \end{minipage}  
            \hfill % 填充空白,使得两个图像并排显示  
            \begin{minipage}{.48\textwidth} % 右侧图像占文本宽度的48%  
                \centering  
                \includegraphics[width=\linewidth]{残差-频率直方图.png}  
                \caption{残差-频率直方图}  
                \label{res}
            \end{minipage}  
        \end{figure}
        
        
        基于此,本文通过第\pageref{I}页的式(\ref{I})可转换得到电载量与环境温度和弧垂角度的定量关系
        \begin{equation}\label{I_ta}
            I=\sqrt[]{ \frac{6901.0}{(1.327\times 10^{-5}\times A + 0.0001356\times T_{e} + 0.02594)} }
        \end{equation}
        
        %%%%%%%%%%%%%%%%%%%%%%%%%%%%%%问题三%%%%%%%%%%%%%%%%%%%%%%%%%%%%%%
	\section{问题三模型的建立与求解}
	%模型建立是将原问题抽象成用数学语言的表达式,它一定是在先前的问题分析和模型假设的基础上得来的。因为比赛时间很紧,大多时候都是使用已经建立好的模型。这部分一定要将题目问的问题和模型紧密结合起来,切忌随意套用模型。还可以对已有模型的某一方面进行改进或者优化,或者建立不同的模型解决同一个问题,这样就是论文的创新和亮点。
        \subsection{数据预处理}
        \label{model_3}
        由本文在第\pageref{p3_ana}页的分析知,本文需求解应力与弧垂角度的具体线性关系,进而可以得到电流变化与应力变化间的关系。经观察,不同设备测得的应力数据间难以比较,且数据点较为分散(可参考图(\ref{data_begin}-\ref{data_end})),并会随时间推进产生不可比误差。为此,本文对数据做以下处理:
        
        \textbf{1)依据题目所给张力标准偏差按1/10的比例筛除偏差较大点。}

        \textbf{2)引入基于密度的LOF局部异常因子筛除离群点,实现更好的线性拟合效果。}

        \textbf{3)对不同设备数据分别建立线性模型,隔离分析。}

        \textbf{4)选取较为稳定的2019年7月份数据为代表拟合应力与弧垂角度的线性函数。}

        \textbf{5)选取稳定性高的设备二测量的弧垂角度作为本题建模使用的唯一弧垂角度。}

        虽然应力数据随时间推进会产生不可比误差,但应力与弧垂角度的线性相关性并不会改变,且误差在较小时间范围内对线性方程系数的影响有限。因此本文通过选取一个数值稳定的小区间来拟合线性方程的做法是合理的。由于设备一、二测量弧垂角度时所处环境不同,二者数据不可比,故本文选择稳定性较高的设备二测量数据作为本题建模使用的唯一弧垂角度数据(设备二的数据稳定性依据第一题的模型训练损失)。

        本文引入的LOF局部异常因子算法通过计算一个样本点周围的样本的所处位置的平均密度相对于该样本点所在位置的密度的比值来衡量样本点的离群程度。为计算LOF距离,本文先引入如下定义:

        \textbf{1)k邻近距离:}其代表在距离数据点P最近的几个点中,第k个最近的点跟点P之间的距离,定义了LOF算法中“局部”的概念,公式为
        \begin{align}
            d_{k}(P)=d(P,O)
        \end{align}
        
        \textbf{2)k距离邻域:}即与P距离小于等于$d_{k}(P)$的圆形范围,公式为
        \begin{align}
            N_{k}(P)=\left \{d(P,O^{\prime })\le d_{k}(P)\right \}
        \end{align}

        \textbf{3)可达距离:}即O的K邻近距离与OP实际距离中的较大值,公式为
        \begin{align}
            reach\_dist_{k}(O, P)=\max \left\{d_{k}(O), d(O, P)\right\}
        \end{align}

        \textbf{4)局部可达密度:}即平均可达距离倒数,公式为
        \begin{align}
            lrd_{k}(P)=\frac{1}{\frac{\sum_{O \ni N_{k}(P)} r e a c h \_  dist_{k}(P, O)}{\left|N_{k}(P)\right|}}
        \end{align}

        由上述定义本文可以得到LOF局部异常因子的计算公式为式(\ref{calc_lof}),它在数值上等于点P的k邻域内的点的平均局部可达密度与点P的局部可达密度的比值
        \begin{align}
            LOF_{k}(P)=\frac{\sum_{O \ni N_{k}(P)} \frac{lrd(O)}{lrd(P)} }{\left|N_{k}(P)\right|}
            \label{calc_lof}
        \end{align}

        在完成标准偏差较大点的筛除后,本文将不同设备数据(共8组)分别带入LOF计算公式。本文设公式中k值为50。本文以2019年7月份数据为代表,绘制散点图可得图(\ref{data_begin}-\ref{data_end})(见附录A第二小节),其中橙色点为离群点。剔除离群点后本文得到了更优的数据。
       
        
        \subsection{模型的建立}
        \subsubsection{损失函数的选择}
        在本题数据十分离散且异常点较多的情况下的损失函数选择是需慎重的。虽经前一步预处理已得到性质较优的数据,但本文认为根据八类应力数据分布选择不同的损失函数依旧必要。

        本文在第\pageref{model_def}页探讨了基于均方根误差对数变换的损失函数,其在处理非正态分布数据时尤为使用,本文选择它作为钢架腿4通道1、2数据的损失函数。其计算公式如下
        \begin{align}
            \;\sqrt{\frac{1}{n}\sum_{i=1}^{n} (\ln{y_i} -\ln{\hat{y_i} })^{2} }\;
        \end{align}

        对于钢架腿1通道2和钢架腿2通道1数据,由于残存离群点较多,本文为减少残存离群点对模型拟合效果的影响,选择均方根误差作为损失函数。均方根误差通过取根号操作降低模型对于偏离点的损失。

        而对于钢架腿1通道1、钢架腿2通道2、钢架腿3通道1、钢架腿3通道2数据,由于线性关系明显,离群点少,本文取平方误差作为损失函数。这样即降低计算复杂度,也利于模型将所求直线约束在正态分布的中心值区域。

        \subsubsection{模型算法的选择}
        由于本题需要根据八类不同数据作八次训练,本文依旧选择自适应修改学习率的梯度下降法作为模型算法,为了获得更快的收敛速度。具体公式推导见第\pageref{grad}页。

        \subsubsection{模型参数初始化}
        本文在此处的模型初始化参数如表(\ref{tab:model_param2})所示。经试验,该初始化参数在本题八类应力数据上都取得了不错的效果。
        \vspace{16pt}
        \begin{table}[h] 
            \centering
            \caption{模型初始化参数} 
            \begin{tabular}{ccccc}  
                \toprule  
                    学习率 & 最大迭代次数 & 线性函数系数 & 最大误差限\\  
                \midrule  
                    $10^{-7}$ & 1000 & $\begin{bmatrix}0\\0\end{bmatrix}$ & $10^{-8}$ \\ 
                \bottomrule
            \end{tabular}  
            \label{tab:model_param2}
        \end{table}
        
	\subsection{模型的训练}
        \label{model_train}
        将八类应力数据分别投入对应模型进行训练,得到损失函数曲线(图(\ref{loss_begin}-\ref{loss_end}),见附录)。虽然由于残余离群点的存在,曲线在迭代次数较低时震荡较明显,但曲线最终都取得了较好的收敛效果。根据模型拟合结果可绘出应力$F_{\text{应}}$与弧垂角度$A$的线性关系,如图(\ref{p3_result})
        \begin{figure}[H]
		\centering
		\includegraphics[width=.7\textwidth]{p3_result.png}
		\caption{钢架腿应力-弧垂角度线性关系}
		\label{p3_result}     %创立一个标签,上下文可以直接引用其中
	\end{figure}
	
	\subsection{模型的评估}
        本文依旧使用拟合优度来评估模型的拟合效果。由本文\pageref{R2}页式(\ref{R2})可计算得上述八个模型的拟合优度分别为
        \vspace{16pt}
        \begin{table}[h] 
            \centering
            \caption{八种模型的拟合优度} 
            \begin{tabular}{cccc}  
                \toprule  
                    钢架腿1通道1 & 钢架腿1通道2 & 钢架腿2通道1 & 钢架腿2通道2\\  
                \midrule  
                    963.5634 & 739.8385 & 2.4477$\times10^{3}$ & 2.5207$\times10^{3}$\\
                \midrule
                    钢架腿3通道1& 钢架腿3通道2& 钢架腿4通道1& 钢架腿4通道2\\
                \midrule
                	2.4508$\times10^{3}$&2.4418$\times10^{3}$ &500.3269&2.4756$\times10^{3}$\\
                \bottomrule
            \end{tabular}  
            \label{tab:p3_R2}
        \end{table}

        由拟合优度算法可知,模型具有较好的表现力。

        \subsection{函数关系转换}
        设钢架腿序号为i,通道序号为j,则钢架腿i通道j模型对应的参数向量表示为
        \begin{align}
            W_{ij}=\begin{bmatrix}p_{ij1}\\p_{ij2}\end{bmatrix}
        \end{align}
        弧垂角度与钢架腿i通道j的应力的线性关系为
        \begin{align}
            A=F_{\text{应}}\cdot W_{ij}
            \label{AF}
        \end{align}

        又由第\pageref{I_ta}页公式(\ref{I_ta})可知电载量与环境温度和弧垂角度的关系。将公式(\ref{AF}、\ref{I_ta})联立,得导线电流变化(电载量变化)与钢架腿部每侧应力变化之间的关系为
        \begin{align}
            I=\sqrt[]{ \frac{6901.0}{(1.327\times 10^{-5}\times (F_{\text{应}}\cdot W_{ij}) + 0.0001356\times T_{e} + 0.02594)} }
            \label{IF}
        \end{align}

        由公式(\ref{IF})知,在假设环境温度不变的情况下,载电量随应力的增大而减小,而由于应力前的系数不大,故载电量变化幅度较小。

        %%%%%%%%%%%%%%%%%%%%%%%%%%%%%%问题四%%%%%%%%%%%%%%%%%%%%%%%%%%%%%%
	\section{问题四模型的建立与求解}
 
        \subsection{模型建立}
        根据上文\textbf{5.1 电载量与导体温度的关系}的具体分析,可知在稳态热平衡假设下,最大的理论载流量与导线的温度有着较强的依赖关系。在实际情况中,110kV架空输电线路导线的温度并非一个固定的数值,而是受到多种因素的影响,包括导线的材质、截面积、环境温度、风速、日照强度等。对于本例的110kV线路,若使用的是常见的钢芯铝绞线(ACSR)等导线,在正常运行情况下,导线温度应在$-5^\circ$C至$+90^\circ$C之间,当导线温度超过这个范围时,应采取相应的措施,如加强巡检、调整负荷等,以确保导线温度处于安全范围。
        
        具体到本问题,本文先通过使用问题一中针对导线温度预测的LSTM模型预测出2020年8月份的导线温度大致变化情况,如下图所示:

        \begin{figure}[htbp] 
                \centering  
                \includegraphics[width=\linewidth]{8月份导线温度预测.png}  
                \caption{2020年8月份导线温度预测图}
        \end{figure}

        \subsection{模型计算}
        根据此图,可以推测出8月份导线的温度会动态分布在$13^\circ$C附近。
        并且可以根据公式(\ref{I})推算出此条件下8月份的动态载流量情况,如下图所示。
        该动态载流量情况就是所求的最大增容量的极限。
        对这组数据求得平均值为:478.3321$A$。

        \begin{figure}[htbp] 
                \centering  
                \includegraphics[width=.99\linewidth]{8月份导线载流量预测.png}  
                \caption{2020年8月份导线载流量预测图}
        \end{figure}

        
        现在讨论在有了该最大增容量的先验条件后,应如何制定合理的增容方案:

        上文的公式(\ref{I_ta})表明了载流量与环境温度和弧垂角度的定量关系,
        在实际情况下,本文是无法干预环境温度的,只能进行预测,
        在这里本文同样使用LSTM模型(使用表格中的环境温度数据进行训练)进行预测,
        得出8月份的环境温度情况,如下图所示。
        对这组数据求得平均值为:$18.9827^\circ$C。

        \begin{figure}[htbp] 
                \centering  
                \includegraphics[width=.99\linewidth]{8月份环境温度预测.png}  
                \caption{2020年8月份环境温度预测图}
        \end{figure}        
        
        虽然环境温度无法干预,只能预测,但弧垂角度却可以人为调控。
        在电力输电过程中,架空线路的弧垂是指电缆或导线与地面的垂直距离。
        合理的弧垂是确保输电线路正常运行和安全的重要因素。
        然而,由于自然灾害、设备老化等原因,导致弧垂异常过大或过小,
        都会对线路性能和安全性产生负面影响。
        此时,就需要对架空线路进行调整弧垂的工作。

        具体到本题,首先根据上文的公式(\ref{I_ta})可以推出弧垂角度与载流量和环境温度的关系,如下:
        \begin{equation}
            A = \frac{52030150.973}{I^2} - 1.021854 \times T_e - 195.47852
        \end{equation}

        将之前已经计算得到的最大载流量$I=478.3321A$和环境温度$T_e=18.9827^\circ$C代入,
        得:弧垂角度A的值为$12.5268^\circ$,
        这个数值就是为了得到尽可能大的载流量的弧垂角度最小值,
        若将弧垂角度调节到比该值还要小,则很可能会导致导线电流大于导线的载流量,产生危险。

        \subsection{模型运用}
        现在来说明一下调节弧垂角度的具体做法:
        先进行实地勘察,对线路的结构和负荷情况进行综合分析,
        然后使用相应的导线应力弧垂计算软件分析现有弧垂对线路的影响,
        再根据相关规程和标准,制定相应的调整方案。
        调整方案需考虑到线路的安全性、可靠性和经济性等因素,并与线路所在地的电网规划和发展规划相协调。
        再根据调整方案,制定相应的设备采购计划,采用吊装设备将线路的导线适当升高或降低,以调整其弧垂。
        完成线路调整工作后,还需对线路进行参数监测,包括弧垂、温度、电流等。
        使用现场测试仪器和检测计算软件对线路进行全面检测,以确保调整后的弧垂达到要求,
        并且线路的性能和安全性符合标准。

        除了调节弧垂角度之外,还可以应用以下方案:
        \begin{itemize}
            \item 应用动态线路额定(Dynamic Line Rating, DLR)技术:
                  DLR技术能根据实时的环境条件(如温度、风速和日照)调整线路的安全运行容量。这种方法通过实时监测环境数据和线路状态,动态调整输电线路的容量,从而实现在保证安全的前提下最大化输电效率。
            \item 增加导线截面积:
                  通过增加输电线路的导线截面积,可以降低电阻,从而提高输电容量。这种方法适用于长期的增容需求,但施工周期较长,成本相对较高。
            \item 使用高温超导材料:
                  应用新型的高温超导材料替换现有的输电线路。这些材料在高温下仍能保持低电阻,可以显著提升线路的传输能力。
            \item 优化运行模式:
                  通过优化输电线路的运行模式,例如调整电网的负载分配,实现对重载线路的负担减轻。
        \end{itemize}
    

        \newpage
	\section{模型的评价与推广}
	%注:本部分的标题需要根据内容进行调整,例如:如果没有写模型推广的话,就直接把标题写成模型的评价与改进,或直接统称为“模型评价”部分,也是可以的。
	\subsection{模型的优缺点}
	%优缺点是必须要写的内容,改进和推广是可选的,这部分对于整个论文的作用在于画龙点睛。
	\subsubsection{模型的优点}
	\begin{enumerate}
        \item 本文充分考虑到附件提供的时间序列数据时间跨度大,总体数据多,所以使用归一化器 $MinMaxScaler$ 将输入的数据进行归一化处理后转换为$DataFrame$格式,并忽略缺失值,这些处理使数据更加优质且方便后续处理
        \item 本文充分考虑到数据可能存在的多重共线性,进行了MTC检查,确保了各自变量之间不存在多重共线性
        \item 本文充分考虑到长序列时容易发生的梯度消失或梯度爆炸的问题,使用LSTM引入一种特殊的记忆单元,使得模型在不损失长期记忆的情况下更好地处理序列数据,并实现了对长期依赖关系的有效捕获
        \item 本文充分考虑到传统线性回归复杂度较高、模型收敛较慢且有可能梯度爆炸的问题,引入了自适应修改学习率的基于差分的梯度下降法,加快模型收敛并有效解决了梯度爆炸的问题,提高了参数稳定性
        \item 由于本文数据中专家经验数据较少,本文没有使用依赖大量专家经验的孤立森林和局部离群因子算法而使用了更适合本文的$Cook's$距离来衡量数据点的离群程度。
		\item 模型适用于参数多而数据结构复杂,受多因素影响的架空线路载流量预测,可以从弧垂角度、环境温度、导体温度等方面分析载流量情况,对把握影响载流量变化重要因素有重大意义。
		\item 模型的短期预测结果与实际数据吻合度高,泛化程度好,且各因素对载流量的影响情况符合实际。
        \item 本文提出的增容方案基于问题一、二、三所建立的模型,认为合理调控弧垂角度是动态增容的关键;并且本文还提出了其他一些可行的增容方案,具有很强的现实意义。
		
	\end{enumerate}
	\subsubsection{模型的缺点}
	%缺点一定比优点少!
	\begin{enumerate}
		\item 测量设备和数据较少,无法通过传统的基于大规模数据特征呈正态分布的统计原理获得和其他影响因素无关的一般规律,也使得本文建立的模型虽然在小批量数据上表现较好,但由于缺乏大批量数据验证,无法保证当有大量数据需要预测时模型仍有较强的泛化能力。
		\item 本文忽视了电压、太阳辐射、风、湿度、经纬度和海拔高度造成的影响,使得本文建立的模型与现实有一定偏差。
		\item 在预测载流量变化情况时,本文对原始数据进行了一些处理,如归一化处理等,这些方法给最终的预测结果带来了一定的误差。
        \item 本题数据较为温和,但在现实生产中会出现极端环境,本文无法确定模型在极端条件下的表现。
	\end{enumerate}
	
	\subsection{模型的改进与推广}

	\subsubsection{模型的改进}
            \begin{itemize}
                \item 如果有更多的测量设备和监测数据,当数据规模足够大后,可以使本模型在大批次数据中进行训练,使得模型在对预测大量数据时准确率更高。
                \item 如果考虑更多因素,比如在预测载流量的时候考虑湿度、电压对其的影响,所预测的载流量值会更准确,模型的泛化性能会更好。
            \end{itemize}
	
	\subsubsection{数据的推广}
	用本文建立的模型可用于定量分析环境温度、导体温度、弧垂角度、钢架腿部每侧应力对架空线路载流量的影响,为社会众多领域,如输电线路建设,架空线路防护和危害预警、用电分配、工业设施管理和优化运营等方面的发展规划提供必要的信息。同时通过该模型可以较为准确地预测未来的负载需求,从而有利于动态增容调控。

	\subsection{对动态增容调控的一些建议}
        动态增容(Dynamic Line Rating, DLR)是一种高效利用输电线路的方法,通过根据实时天气和环境数据动态调整输电线路的载荷能力。
        
        以下是本文的一些对动态增容调控的建议:
        \begin{enumerate}
            \item  加强实时数据监控和分析,
        强化实时环境监测设备的部署,包括温度、湿度、风速、风向和太阳辐射等参数的监测;
        利用高级数据分析和机器学习算法,对收集到的数据进行实时分析,预测输电线路的最大安全载荷。
            \item  优化监测设备和技术,
        使用更高精度和可靠性的监测设备,以确保数据的准确性和稳定性;
        定期更新和维护监测设备,确保其在恶劣天气条件下也能正常工作。
            \item 制定应急响应机制,
        建立一套完整的应急响应和事故预防机制,以应对突发的环境变化和设备故障;
        在监测到极端天气条件或其他可能导致输电线路过载的情况时,迅速采取措施降低负荷或调整电网配置。
            \item 提高系统的灵活性和适应性,
        通过动态调整输电线路的容量,不仅要考虑当前的环境条件,还要预测未来几小时甚至几天的环境变化;
        实施更为灵活的电网管理和调度策略,以适应不断变化的输电需求和环境条件
            \item 进行持续的技术创新和研究,
        投资于与高校和研究机构的合作,推动DLR技术的创新和进步;
        探索与其他智能电网技术(如储能系统和可再生能源)的集成,以进一步提高电网的稳定性和效率。
            \item 提升操作人员的专业培训,
        对电网操作人员进行DLR技术和相关软件工具的定期培训,确保他们能够有效地管理和利用这些先进技术;
        增强团队对于数据分析和环境监测的理解,使他们能够更好地做出决策。
        \end{enumerate}
        通过对载流量动态增容调控,国家可以高效利用架空线路,减少输电过程中由各种因素造成的浪费和损耗,有利于最大化利用电力资源,具有非常大的现实意义。

	%%%%%%%%%%%%%%%%%%%%%%%%%%%%%%参考文献%%%%%%%%%%%%%%%%%%%%%%%%%%%%%%
	\begin{thebibliography}{99}  
        \bibitem{ref1} Vanderweide L A , Foster C J , Maclaren R ,et al.738-1993 - IEEE Standard for Calculating the Current-Temperature of Bare Overhead Conductors[J].IEEE, 2002.DOI:10.1109/IEEESTD.1993.120365.
		\bibitem{ref2} 刘士璋. 铝绞线钢芯铝绞线交直流电阻及载流量的计算. 上海电缆研究所.上海. 200093
	\end{thebibliography}

 
	\newpage
 
	%%%%%%%%%%%%%%%%%%%%%%%%%%%%%%附录%%%%%%%%%%%%%%%%%%%%%%%%%%%%%%
	\begin{appendices}
		
		\section{文中部分图片}
        \subsection{时间序列分解OTLM1电流数据结果图}
        以下图片用于本文第\pageref{model_1}页分析不同时间间隔下电流(动态载流量需求)的周期性特征。         
        \begin{figure}[h!] 
            \begin{minipage}{.45\textwidth} % 左侧图像占文本宽度的48%  
                \centering  
                \includegraphics[width=\linewidth]{D1_0.5h.png}  
                \caption{时间间隔30min\_时间序列分解图}  
            \end{minipage}  
            \hfill % 填充空白,使得两个图像并排显示  
            \begin{minipage}{.45\textwidth} % 右侧图像占文本宽度的48%  
                \centering  
                \includegraphics[width=\linewidth]{D1_1h.png}  
                \caption{时间间隔1h\_时间序列分解图}  
            \end{minipage}  
        \end{figure}
        
        \begin{figure}[h!] 
            \begin{minipage}{.45\textwidth} % 左侧图像占文本宽度的48%  
                \centering  
                \includegraphics[width=\linewidth]{D1_2h.png}  
                \caption{时间间隔2h\_时间序列分解图}  
            \end{minipage}  
            \hfill % 填充空白,使得两个图像并排显示  
            \begin{minipage}{.45\textwidth} % 右侧图像占文本宽度的48%  
                \centering  
                \includegraphics[width=\linewidth]{D1_4h.png}  
                \caption{时间间隔4h\_时间序列分解图}  
            \end{minipage}  
        \end{figure}

        \begin{figure}[h!] 
            \begin{minipage}{.45\textwidth} % 左侧图像占文本宽度的48%  
                \centering  
                \includegraphics[width=\linewidth]{D1_8h.png}  
                \caption{时间间隔8h\_时间序列分解图}  
            \end{minipage}  
            \hfill % 填充空白,使得两个图像并排显示  
            \begin{minipage}{.45\textwidth} % 右侧图像占文本宽度的48%  
                \centering  
                \includegraphics[width=\linewidth]{D1_16h.png}  
                \caption{时间间隔16h\_时间序列分解图}  
            \end{minipage}  
        \end{figure}

        \begin{figure}[h!] 
            \begin{minipage}{.45\textwidth} % 左侧图像占文本宽度的48%  
                \centering  
                \includegraphics[width=\linewidth]{D1_32h.png}  
                \caption{时间间隔32h\_时间序列分解图}  
            \end{minipage}  
            \hfill % 填充空白,使得两个图像并排显示  
            \begin{minipage}{.45\textwidth} % 右侧图像占文本宽度的48%  
                \centering  
                \includegraphics[width=\linewidth]{D1_64h.png}  
                \caption{时间间隔64h\_时间序列分解图}  
            \end{minipage}  
        \end{figure}

  
		\subsection{四条钢架腿的样本分布及离群点可视化散点图}
            以下图片用于本文第\pageref{model_3}页通过LOF局部异常因子算法预处理问题三模型的数据。  
             \begin{figure}[h!] 
            \begin{minipage}{.48\textwidth} % 左侧图像占文本宽度的48%  
                \centering  
                \includegraphics[width=\linewidth]{11.png}  
                \caption{钢架腿1通道1}  
                \label{data_begin}
            \end{minipage}  
            \hfill % 填充空白,使得两个图像并排显示  
            \begin{minipage}{.48\textwidth} % 右侧图像占文本宽度的48%  
                \centering  
                \includegraphics[width=\linewidth]{12.png}  
                \caption{钢架腿1通道2}  
            \end{minipage}  
        \end{figure}
        \begin{figure}[h!] 
            \begin{minipage}{.48\textwidth} % 左侧图像占文本宽度的48%  
                \centering  
                \includegraphics[width=\linewidth]{21.png}  
                \caption{钢架腿2通道1}  
            \end{minipage}  
            \hfill % 填充空白,使得两个图像并排显示  
            \begin{minipage}{.48\textwidth} % 右侧图像占文本宽度的48%  
                \centering  
                \includegraphics[width=\linewidth]{22.png}  
                \caption{钢架腿2通道2}  
            \end{minipage}  
        \end{figure}
        \begin{figure}[h!] 
            \begin{minipage}{.48\textwidth} % 左侧图像占文本宽度的48%  
                \centering  
                \includegraphics[width=\linewidth]{31.png}  
                \caption{钢架腿3通道1}  
            \end{minipage}  
            \hfill % 填充空白,使得两个图像并排显示  
            \begin{minipage}{.48\textwidth} % 右侧图像占文本宽度的48%  
                \centering  
                \includegraphics[width=\linewidth]{32.png}  
                \caption{钢架腿3通道2}  
            \end{minipage}  
        \end{figure}
        \begin{figure}[h!] 
            \begin{minipage}{.48\textwidth} % 左侧图像占文本宽度的48%  
                \centering  
                \includegraphics[width=\linewidth]{41.png}  
                \caption{钢架腿4通道1}  
            \end{minipage}  
            \hfill % 填充空白,使得两个图像并排显示  
            \begin{minipage}{.48\textwidth} % 右侧图像占文本宽度的48%  
                \centering  
                \includegraphics[width=\linewidth]{42.png}  
                \caption{钢架腿4通道2}  
                \label{data_end}
            \end{minipage}  
        \end{figure}

        \subsection{四条钢架腿对应模型训练损失曲线}
            以下图片用于本文第\pageref{model_train}页基于八类应力-弧垂角度数据训练得到的损失曲线。  
             \begin{figure}[h!] 
            \begin{minipage}{.48\textwidth} % 左侧图像占文本宽度的48%  
                \centering  
                \includegraphics[width=\linewidth]{loss_11.png}  
                \caption{钢架腿1通道1训练损失}  
                \label{loss_begin}
            \end{minipage}  
            \hfill % 填充空白,使得两个图像并排显示  
            \begin{minipage}{.48\textwidth} % 右侧图像占文本宽度的48%  
                \centering  
                \includegraphics[width=\linewidth]{loss_12.png}  
                \caption{钢架腿1通道2训练损失}  
            \end{minipage}  
        \end{figure}
        \begin{figure}[h!] 
            \begin{minipage}{.48\textwidth} % 左侧图像占文本宽度的48%  
                \centering  
                \includegraphics[width=\linewidth]{loss_21.png}  
                \caption{钢架腿2通道1训练损失}  
            \end{minipage}  
            \hfill % 填充空白,使得两个图像并排显示  
            \begin{minipage}{.48\textwidth} % 右侧图像占文本宽度的48%  
                \centering  
                \includegraphics[width=\linewidth]{loss_22.png}  
                \caption{钢架腿2通道2训练损失}  
            \end{minipage}  
        \end{figure}
        \begin{figure}[h!] 
            \begin{minipage}{.48\textwidth} % 左侧图像占文本宽度的48%  
                \centering  
                \includegraphics[width=\linewidth]{loss_31.png}  
                \caption{钢架腿3通道1训练损失}  
            \end{minipage}  
            \hfill % 填充空白,使得两个图像并排显示  
            \begin{minipage}{.48\textwidth} % 右侧图像占文本宽度的48%  
                \centering  
                \includegraphics[width=\linewidth]{loss_32.png}  
                \caption{钢架腿3通道2训练损失}  
            \end{minipage}  
        \end{figure}
        \begin{figure}[h!] 
            \begin{minipage}{.48\textwidth} % 左侧图像占文本宽度的48%  
                \centering  
                \includegraphics[width=\linewidth]{loss_41.png}  
                \caption{钢架腿4通道1训练损失}  
            \end{minipage}  
            \hfill % 填充空白,使得两个图像并排显示  
            \begin{minipage}{.48\textwidth} % 右侧图像占文本宽度的48%  
                \centering  
                \includegraphics[width=\linewidth]{loss_42.png}  
                \caption{钢架腿4通道2训练损失}  
                \label{loss_end}
            \end{minipage}  
        \end{figure}


        \newpage
		\section{matlab 源代码}
		
		\begin{lstlisting}[language=matlab]
    %%%%%%%%%%%%%%%%%%%%%%%%%%%%%%%%%%%%%%%%%%%%%%%%%%%%%%%%%%%%%%%
    %%%%%%%%%%%%%%%%%%%%%%%% 读取文件 %%%%%%%%%%%%%%%%%%%%%%%%%%%%%
    function data=read_data(device, num_per_unit)
        filename=["2019-7" "2019-8" "2019-9" "2019-10" "2019-11" "2019-12" "2020-2" "2020-3" "2020-4" "2020-5" "2020-6" "2020-7"];
        range_min = 1;
        data=[];
        for f=filename
            data_temp=readmatrix(f+"-cleaned.xlsx");
            if device==1
            data_temp=data_temp(:,[4, 10, 2]);
            elseif device==2
                data_temp=data_temp(:,[5, 11, 3]);
            end
            if nargin==2
            range_max = size(data_temp, 1);
            random_num = fix(range_min + (range_max - range_min+1) * rand([1, num_per_unit]));
            data_temp=data_temp(random_num,:);
            end
            data=vertcat(data, data_temp);
        end
        
    %%%%%%%%%%%%%%% 文件数据故障修复 %%%%%%%%%%%%%%%%%%%%%%%
    filename = ["2019-7" "2019-8" "2019-10" "2019-9" "2019-11" "2019-12" "2020-2" "2020-3" "2020-4" "2020-5" "2020-6" "2020-7"];
    for f = filename
        % 读取Excel文件
        data = readtable(f + "-cleaned.xlsx");
        
        % 指定要处理的列(假设是第十列)
        column_to_process = 11; 
        
        % 循环遍历该列,并对每个单元格进行处理
        for i = 1:height(data)
            % 将单元格的数据转换为double类型
            cell_data = data.(column_to_process)(i);
            
            % 如果数据大于100,则除以1000
            if ~isnan(cell_data) && cell_data > 100
                data.(column_to_process)(i) = cell_data / 1000;
            end
        end
        
        % 写回到Excel文件
        writetable(data, f + "-cleaned.xlsx");
    end
    
    %%%%%%%%%%%%%%%%%% 方差膨胀系数、条件数、Pearson相关系数计算%%%%%%%%%%%%%%%
    R=corr(data_cell1{index}(:,:)); %Pearson相关系数
    VIF=@(X)diag(inv(corr(X))); %方差膨胀系数
    VIF_out1=VIF(data_cell1{index}(:,1:2)); 
    collintest(data_cell1{index}(:,1:2), 'Plot','on');  %条件数

    %%%%%%%%%%%%%%%% 自适应修改学习率的梯度下降法 %%%%%%%%%%%%%%%%%%%%%
    function [beta, loss, iter, fitted, RMSE] = GradDesent_single(X, y, init, eta, maxit, err)
        switch nargin
            case 4
                maxit=1000;
                err=1e-5;
            case 5
                err=1e-5;
        end
        n=size(X,1);
        X=[ones(n,1),X];
        beta=init;
        loss=mean((X*beta-y).^2);
        tol=1;
        iter=1;
        while tol > err && iter<maxit
            fitted=X*beta;
            grad=X'*(fitted-y);
            betaC=beta-eta*grad;
            tol=max(abs(betaC-beta));
            beta=betaC;
            loss=[loss;mean((X*beta-y).^2)];
            iter=iter+1;
            if loss(iter)<loss(iter-1)
                eta=eta*1.2;
            else
                eta=eta*0.8;
            end
        end
        RMSE=sqrt(mean((X*beta-y).^2));

    %%%%%%%%%%%%%%%%%% Cook's距离计算 %%%%%%%%%%%%%%%%%%%
    function cook_distance=CookDistance(X, theta, Y)
        % 计算模型预测值
        X=[ones(size(X,1),1),X];
        y_pred = X*theta;
        
        % 计算残差
        residuals = Y - y_pred;
        
        % 计算Hat矩阵
        H = X / inv(X' * X) * X';
        
        % 计算Cook距离
        n = size(X, 1);
        p = size(X, 2);
        cook_distance = zeros(n, 1);
        for i = 1:n
            cook_distance(i) = residuals(i)^2 / (p * var(residuals) * (p + 1) * sum((residuals.^2) / ((p + 1) * var(residuals))));
        end
        index=1:length(cook_distance);
        scatter(index, cook_distance', 'x');
        hold on;
        standard=3*ones(size(cook_distance,1),1)*mean(cook_distance);
        plot(index, standard, 'r--')
        xlabel('行编号');ylabel('Cook''s距离');title('Cook''s距离-观测顺序图')

    %%%%%%%%%%%%%%%%%%%%% 拟合优度计算 %%%%%%%%%%%%%%%%%%%%%%%%%%%
    function R2=calc_R2(X,theta,Y)
        X=[zeros(size(X,1),1),X];
        fitted=X*theta;
        SST=sum((Y-mean(Y)).^2);
        SSR=sum((fitted-mean(Y)).^2);
        SSE=sum((Y-fitted).^2);
        R2=SSR./SST;
        %R2=1-SSE./SST;

    %%%%%%%%%%%%%%%% LOF局部离散因子计算 %%%%%%%%%%%%%%%%%%%
    function lof = LOF_func(data, k)  
        n = size(data, 1); % 数据点的数量  
        % 计算每个点的k距离  
        k_dist = zeros(n, 1);  
        for i = 1:n  
            distance = sqrt(sum((data - repmat(data(i,:), n, 1)).^2, 2));  
            temp = sort(distance, 'ascend');% 对距离从小到大排序,取第k+1个值 
            k_dist(i)=temp(k+1);
        end  
        % 计算每个点的局部可达密度(LRD)  
        lrd = zeros(n, 1);  
        for i = 1:n  
            neighbors = find(sqrt(sum((data - repmat(data(i, :), n, 1)).^2, 2)) <= k_dist(i));  
            sum_density = sum(k_dist(neighbors));  
            lrd(i) = length(neighbors) / sum_density;  
        end  
        % 计算每个点的LOF值  
        lof = zeros(n, 1);  
        for i = 1:n  
            neighbors = find(sqrt(sum((data - repmat(data(i, :), n, 1)).^2, 2)) <= k_dist(i));  
            lrd_ratio = sum(lrd(neighbors)) / lrd(i);  
            lof(i) = lrd_ratio / length(neighbors);  
        end  

    %%%%%%%%%%%%%% 绘图逻辑 %%%%%%%%%%%%%%%%%
    test_data=readmatrix("2019-7-cleaned.xlsx");
    index=32:5:67;
    %plot(test_data(:,10),test_data(:,index));
    % LOF散点图
    for id=index
        test_data=sortrows(test_data,id+3);
        figure;
        scatter(test_data(1:5800,10),test_data(1:5800,id));
        %title(id);
        hold on
        scores=LOF_func([test_data(1:5800,10),test_data(1:5800,id)],50);
        outliers=find(scores>1.1*mean(scores));
        scatter(test_data(outliers,10),test_data(outliers,id));
        xlabel('弧垂角度(°)');ylabel('钢架腿部应力(N)')
    end

    % Cook's距离视图
    subplot(1,2,1);
    index=1:length(cook_dis1);
    scatter(index, cook_dis1', 'x');
    hold on;
    standard=3*ones(size(cook_dis1,1),1)*mean(cook_dis1);
    plot(index, standard, 'r--')
    xlabel('观测编号');ylabel('设备一Cook''s距离');
    subplot(1,2,2)
    index=1:length(cook_dis2);
    scatter(index, cook_dis2', 'x');
    hold on;
    standard=3*ones(size(cook_dis2,1),1)*mean(cook_dis2);
    plot(index, standard, 'r--')
    xlabel('观测编号');ylabel('设备二Cook''s距离');

    % 频率-残差直方图
    function draw_res_graph(X,theta,Y)
        X=[zeros(size(X,1),1),X];
        fitted=X*theta;
        res=Y-fitted;
        figure;
        histogram(res, 'Normalization', 'probability', 'EdgeColor', 'none');
        xlabel('残差值');
        ylabel('频率');
   
		\end{lstlisting}

  
		\section{Python 源代码}
		\begin{lstlisting}[language=python]
		#####################################################################################

        ####################### 统一表格中的数据格式 #######################

        import pandas as pd 

        # 文件名列表
        filename = [
            "2019-7.xlsx",
            "2019-8.xlsx",
            "2019-9.xlsx",
            "2019-10.xlsx",
            "2019-11.xlsx",
            "2019-12.xlsx",
            "2020-1.xlsx",
            "2020-2.xlsx",
            "2020-3.xlsx",
            "2020-4.xlsx",
            "2020-5.xlsx",
            "2020-6.xlsx",
            "2020-7.xlsx"
        ]
            
        for f in filename:
            # 读取 Excel 文件
            data = pd.read_excel(f)

            # 将逗号替换为点, 并尽可能将列转换为数字类型
            data = data.apply(lambda x: x.replace(',', '.', regex=True) if x.dtype == 'object' else x)
            data = data.apply(pd.to_numeric, errors='ignore')

            # 获取需要处理的列名列表
            columns_to_process = [col for col in data.columns if col not in ['Date/Time', 'longitude', 'latitude', 'altitude [m]']]
    
            # 循环遍历每一列,并对每个单元格进行处理
            for column_name in columns_to_process:
                for i in range(len(data)):
                    # 将单元格的数据转换为 float 类型
                    cell_data = data.loc[i, column_name]
                    # 如果数据的绝对值大于等于1000,则除以1000
                    if not pd.isnull(cell_data) and abs(cell_data) >= 1000:
                        data.loc[i, column_name] = cell_data / 1000

            # 保存到 Excel 文件
            data.to_excel("result/" + f + "-cleaned.xlsx", index=False)

        ######################### 创建LSTM模型并预测 #########################

        import pandas as pd
        import numpy as np
        from sklearn.preprocessing import MinMaxScaler
        from tensorflow.keras.models import Sequential
        from tensorflow.keras.layers import LSTM, Dense, Dropout
        from keras.optimizers import RMSprop
        from sklearn.model_selection import train_test_split
        import matplotlib.pyplot as plt

        files = [
            "result/2019-7.xlsx-cleaned.xlsx",
            "result/2019-8.xlsx-cleaned.xlsx",
            "result/2019-9.xlsx-cleaned.xlsx",
            "result/2019-10.xlsx-cleaned.xlsx",
            "result/2019-11.xlsx-cleaned.xlsx",
            "result/2019-12.xlsx-cleaned.xlsx",
            "result/2020-1.xlsx-cleaned.xlsx",
            "result/2020-2.xlsx-cleaned.xlsx",
            "result/2020-3.xlsx-cleaned.xlsx",
            "result/2020-4.xlsx-cleaned.xlsx",
            "result/2020-5.xlsx-cleaned.xlsx",
            "result/2020-6.xlsx-cleaned.xlsx",
            "result/2020-7.xlsx-cleaned.xlsx"
        ]
        # 合并所有文件的数据
        all_data = pd.concat([pd.read_excel(file, usecols=["Line Temperature [掳C] DEV1", "Line Temperature [掳C] DEV2"]) for file in files], ignore_index=True)
        # 初始化归一化器
        scaler = MinMaxScaler()
        # 归一化处理
        all_data_scaled = scaler.fit_transform(all_data)
        # 显示归一化器的参数和属性
        print("Feature range:", scaler.feature_range)
        print("Min_:", scaler.min_)
        print("Scale_:", scaler.scale_)
        print("Data min_:", scaler.data_min_)
        print("Data max_:", scaler.data_max_)
        print("Data range_:", scaler.data_range_)
        # 转换为 DataFrame 方便后续操作
        all_data_scaled_df = pd.DataFrame(all_data_scaled, columns=["Line Temperature [掳C] DEV1", "Line Temperature [掳C] DEV2"])

        # 设置时间窗口大小
        time_window = 50
        # 创建数据集
        def create_dataset(data, time_window):
            X, y = [], []
            for i in range(len(data) - time_window):
                end_ix = i + time_window
                if end_ix >= len(data):
                    break
                seq_x, seq_y = data[i:end_ix], data[end_ix]
                X.append(seq_x)
                y.append(seq_y)
            return np.array(X), np.array(y)
        features, targets = create_dataset(all_data_scaled_df[["Line Temperature [掳C] DEV1", "Line Temperature [掳C] DEV2"]].values, time_window)
        # 调整数据维度以适应 LSTM 输入要求:(samples, time steps, features)
        features_reshaped = features.reshape((features.shape[0], features.shape[1], 2))
        # 划分数据为训练集和测试集
        X_train, X_test, y_train, y_test = train_test_split(features_reshaped, targets, test_size=0.2, random_state=42)
        # 检查训练数据和测试数据的形状
        print("Training data shape:", X_train.shape)
        print("Test data shape:", X_test.shape)

        # 初始化 LSTM 模型
        model = Sequential([
            LSTM(100, activation='tanh', return_sequences=True, input_shape=(time_window, 2)),  
            Dropout(0.2),  
            LSTM(50, activation='tanh'), 
            Dense(2)  # 本文预测两个特征
        ])
        # 编译模型, 使用 RMSprop 优化器并调整学习率, 使用均方误差作为损失函数
        optimizer = RMSprop(learning_rate=0.001)
        model.compile(optimizer=optimizer, loss='mean_squared_error')
        # 查看模型结构
        model.summary()   
        # 训练模型
        model.fit(X_train, y_train, epochs=50, batch_size=32, validation_split=0.2, verbose=1)
        # 评估模型
        print("Test loss:", model.evaluate(X_test, y_test))

        # 保存模型
        model.save('lstm_model_3.h5')
    
        def save_and_plot_forecast(forecast, file_name='forecast_7.xlsx'):
            # 创建时间索引
            future_dates = pd.date_range(start='2020-08-01', periods=len(forecast), freq='10min')
            # 将预测数据转换为 DataFrame
            forecast_df = pd.DataFrame(forecast, index=future_dates, columns=['Line Temperature [掳C] DEV1', 'Line Temperature [掳C] DEV2'])
            # 保存到Excel
            forecast_df.to_excel(file_name)
            # 绘图展示
            plt.figure(figsize=(14, 7))
            plt.plot(forecast_df['Line Temperature [掳C] DEV1'], label='Forecast Line Temperature [掳C] DEV1')
            plt.plot(forecast_df['Line Temperature [掳C] DEV2'], label='Forecast Line Temperature [掳C] DEV2')
            plt.title('Forecast for August 2020')
            plt.xlabel('Date')
            plt.ylabel('Line Temperature')
            plt.legend()
            plt.grid(True)
            plt.show()

        # 示例预测数据
        forecast = np.random.rand(31*24*6, 50, 2)  # 假设有31天的数据, 两个电流列
        # 进行预测
        predictions = model.predict(forecast)
        # 反归一化
        predictions = scaler.inverse_transform(predictions)
        # 调用函数来保存数据并绘图
        save_and_plot_forecast(predictions)

        ######################### 拟合线性回归 #########################

        import pandas as pd
        import statsmodels.api as sm

        # 读取数据
        data = pd.read_excel("forecast_102.xlsx")
        # 将其中几列转换为适当的数据类型
        numeric_cols = [
            'Line Current [A] DEV1', 'Line Current [A] DEV2', 
        ]
        for col in numeric_cols:
            data[col] = pd.to_numeric(data[col], errors='coerce')
        # 显示日期/时间和数字列
        print(data.dtypes.head(len(numeric_cols) + 1))

        # 添加常数列,作为线性回归模型中的截距
        X = sm.add_constant([i for i in range(len(data))])
        y = data['Line Current [A] DEV1']

        # 拟合线性回归模型
        model = sm.OLS(y, X).fit()

        # 打印回归结果摘要
        print(model.summary())

        ######################### 傅里叶变换 #########################

        import pandas as pd  
        import numpy as np  
        import matplotlib.pyplot as plt  

        # 读取数据
        data = pd.read_excel("forecast_103.xlsx")
        # 将其中几列转换为适当的数据类型
        numeric_cols = [
            'Line Current [A] DEV1', 'Line Current [A] DEV2', 
        ]
        for col in numeric_cols:
            data[col] = pd.to_numeric(data[col], errors='coerce')

        # 将日期设置为索引
        data['Date/Time'] = pd.to_datetime(data['Date/Time'], format='%Y/%m/%d %H:%M:%S', errors='coerce')
        # data['Date/Time'] = pd.to_datetime(data['Date/Time'], format='%d.%m.%Y. %H:%M:%S', errors='coerce')
        # data['Date/Time'] = pd.date_range(start="2020-08-01", periods=4707, freq='9min')
        data.set_index('Date/Time', inplace=True)
        # 设置频率, 重采样(使用 mean 作为聚合函数)或者使用其他聚合函数, 如sum, min, max等
        resampled_data = data.resample('4D').max()    

        # 计算了 resampled_data 的长度, 即数据点的数量
        n = len(resampled_data)  
        # 取出数据
        y = resampled_data['Line Current [A] DEV2'].values  
  
        # 执行实数快速傅里叶变换 RFFT
        yf = np.fft.rfft(y)  # yf 是一个数组, 包含了 FFT 的结果, 这个数组的每个元素都代表了一个频率分量的幅度和相位
        xf = np.fft.rfftfreq(n, d=1/n)  # xf 是一个数组, 包含了与 yf 中每个 FFT 结果对应的频率值
   
        # 绘制频谱图  
        plt.figure(figsize=(12, 6))  
        plt.plot(xf, np.abs(yf), 'r')  # 绘制 FFT 的绝对值  
        plt.xlabel('Freq (Hz)')  
        plt.ylabel('|Y(freq)|')  
        plt.title('FFT - Magnitude Spectrum')  
        plt.grid(True)  
        plt.show()  
            
        ######################### 时间序列分解 #########################

        import pandas as pd

        # 读取数据
        data = pd.read_excel("forecast_102.xlsx")
        # 将其中几列转换为适当的数据类型
        numeric_cols = [
            'Line Current [A] DEV1', 'Line Current [A] DEV2', 
        ]
        for col in numeric_cols:
            data[col] = pd.to_numeric(data[col], errors='coerce')
        # 显示日期/时间和数字列
        print(data.dtypes.head(len(numeric_cols) + 1))

        import matplotlib.pyplot as plt
        from statsmodels.tsa.seasonal import seasonal_decompose

        # 将日期设置为索引
        data['Date/Time'] = pd.to_datetime(data['Date/Time'], format='%Y/%m/%d %H:%M:%S', errors='coerce')
        # data['Date/Time'] = pd.to_datetime(data['Date/Time'], format='%d.%m.%Y. %H:%M:%S', errors='coerce')
        # data['Date/Time'] = pd.date_range(start="2020-08-01", periods=4707, freq='9min')
        data.set_index('Date/Time', inplace=True)

        # 设置频率, 重采样(使用 mean 作为聚合函数)或者使用其他聚合函数, 如sum, min, max等
        data_resampled = data.resample('64h').mean()    
        # 设置频率
        # freq = pd.infer_freq(data.index)  
        # data.index.freq = freq
        # 提取线路电流[A] DEV1 进行分析(现在从重采样后的数据集中提取) 
        current_data = data_resampled['Line Current [A] DEV1'].dropna()
        # 提取线路电流[A] DEV1 进行分析
        # current_data = data['LineCurrent_A_DEV1'].dropna()
        # 分解时间序列以观察趋势、季节性和残差
        decomposition = seasonal_decompose(current_data, model='additive', period=int(31 * 3 / 16))
        # 绘制分解图
        plt.figure(figsize=(14, 7))
        decomposition.plot()
        plt.show()

        ########################## 绘曲线图 ##########################

        import pandas as pd
        import matplotlib.pyplot as plt

        # 读取数据
        data = pd.read_excel("forecast_103.xlsx")
        # data = pd.read_excel("forecast_1.xlsx", usecols=[1, 2])
        # 将其中几列转换为适当的数据类型
        numeric_cols = [
            'Line Current [A] DEV1', 'Line Current [A] DEV2', 
        ]
        for col in numeric_cols:
            data[col] = pd.to_numeric(data[col], errors='coerce')

        # 将日期设置为索引
        data['Date/Time'] = pd.to_datetime(data['Date/Time'], format='%Y/%m/%d %H:%M:%S', errors='coerce')
        # data['Date/Time'] = pd.to_datetime(data['Date/Time'], format='%d.%m.%Y. %H:%M:%S', errors='coerce')
        # data['Date/Time'] = pd.date_range(start="2020-08-01", periods=4707, freq='9min')
        data.set_index('Date/Time', inplace=True)

        # 重采样(使用mean作为聚合函数)或者使用其他聚合函数,如sum, min, max等
        data = data.resample('6h').mean()    

        # 绘制其中一列电流数据
        plt.plot(data['Line Current [A] DEV1'], label='Line Current [A] DEV1', marker='o')

        # 设置图表标题和图例
        plt.title('Forecast for August 2020')
        plt.xlabel('Date Time')
        plt.ylabel('Line Current [A] DEV1')
        plt.legend()
        plt.grid(True)
        plt.show()
        
        #####################################################################################
		\end{lstlisting}
	\end{appendices}
	
\end{document}

